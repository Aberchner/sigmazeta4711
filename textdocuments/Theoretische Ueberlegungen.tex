\documentclass[11pt]{scrartcl}
\begin{document}
	\section{Martingale und Varianten}
		\subsection{Einführung in Martingale}
			Man betrachte zunächst ein Spiel bei dem auf Sieg und Niederlage gewettet werden kann. Die Wahrscheinlichkeit für einen Sieg sei also \(p_{0}\) und eine Niederlage \(p_{1}\) (Meist gilt hier \(p_{0} = 1-p_{1}\)). Am Anfang des Spiels legt sich der Spieler auf Sieg oder Niederlage fest. Sei dies hier o.B.d.A. Sieg. In der ersten Runde setzt der Spieler also einen Betrag \(x_{1}\)  auf Sieg. Bei einer Niederlage wird wiederholt auf Sieg gesetzt und der Betrag vervielfacht, d.h. \(\alpha x_{i} = x_{i+1}\). In den bekanntesten Varianten wird hier \(\alpha = 2\) gewaehlt. Sobald auf einer Stufe ein Sieg eintritt, wird in der nächsten Runde wieder mit \(x_{0}\) gestartet. \(\alpha\) sollte also so gewählt sein das es möglich ist auf jeder Stufe durch einen Sieg den Verlust der vorherigen Stufen auszugleichen.
		\subsection{Verallgemeinertes Martingale}
			In diesem Abschnitt soll Martingale auf (Aktien-)Kurse angewendet und verallgemeinert werden. Dafuer seien im Folgendem \(p_{1}, p_{2}, ...,p_{n}\) die Wahrscheinlichkeiten fuer das steigen eines Kurses. Des weiteren seien \(x_{1}, x_{2}, ...,x_{n}\) die jeweiligen Einsaetze, \(\gamma\) der Gewinnfaktor (d.h. der Gewinn bei einem Sieg auf der i-ten Stufen beträgt \(\gamma x_{i}\)) und \(\alpha_{1}, \alpha_{2}, ...,\alpha_{n}\) die Steigerungsfaktoren (d.h. \(\alpha_{i}x_{i}=x_{i+1}\)).\\\\
			D.h. der Gewinn auf der i-ten Stufe entspricht entweder \(-x_{i}\) oder \(\gamma x_{i}\). Sei also \(\psi(p_{i})\) dieser Gewinn. Dann gilt fuer den Erwartungswert
			\begin{center}
				\(E(X)=\sum_{i=1}^n \psi(p_{i})p_{i}\)
			\end{center}
			Und fuer jeden Einsatz gilt
			\begin{center}
				\(x_{k}=\alpha_{k-1}x_{k-1}=\alpha_{k-1}\alpha_{k-2}x_{k-2}=...=\prod_{i=1}^{k-1}\alpha_{i}x_{1}\)
			\end{center}
			Sei \(\mu\) der maximale zur Verfuegung stehende Betrag, d.h
			\begin{center}
				\(\mu>= \sum_{i=1}^{n}x_{i}=\sum_{i=1}^{n}\prod_{k=1}^{i-1}\alpha_{k}x_{1}\)
			\end{center}
			Das Ziel ist jetzt also den Erwartungswert nach den Einsaetzen zu optimieren und nach Strategien in Abhaengigkeit von den Gewinnwahrscheinlichkeiten zu finden.
\end{document}