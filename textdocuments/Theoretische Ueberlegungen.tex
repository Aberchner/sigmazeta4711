\documentclass[11pt]{scrartcl}
\usepackage{amsmath}
\usepackage{amssymb}
\begin{document}
	\section{Martingale und Varianten}
		\subsection{Einführung in Martingale}
			Man betrachte zunächst ein Spiel bei dem auf Sieg und Niederlage gewettet werden kann. Die Wahrscheinlichkeit für einen Sieg sei also \(p_{0}\) und eine Niederlage \(p_{1}\) (Meist gilt hier \(p_{0} = 1-p_{1}\)). Am Anfang des Spiels legt sich der Spieler auf Sieg oder Niederlage fest. Sei dies hier o.B.d.A. Sieg. In der ersten Runde setzt der Spieler also einen Betrag \(x_{1}\)  auf Sieg. Bei einer Niederlage wird wiederholt auf Sieg gesetzt und der Betrag vervielfacht, d.h. \(\alpha x_{i} = x_{i+1}\). In den bekanntesten Varianten wird hier \(\alpha = 2\) gewaehlt. Sobald auf einer Stufe ein Sieg eintritt, wird in der nächsten Runde wieder mit \(x_{0}\) gestartet. \(\alpha\) sollte also so gewählt sein das es möglich ist auf jeder Stufe durch einen Sieg den Verlust der vorherigen Stufen auszugleichen.
		\subsection{Verallgemeinertes Martingale}
			In diesem Abschnitt soll Martingale auf (Aktien-)Kurse angewendet und verallgemeinert werden. Dafuer seien im Folgendem \(p_{1}, p_{2}, \cdots,p_{n}\) die Wahrscheinlichkeiten fuer das steigen eines Kurses. Des weiteren seien \(x_{1}, x_{2}, \cdots,x_{n}\) die jeweiligen Einsaetze und X der zugehoerige Vektor, \(\gamma\) der Gewinnfaktor (d.h. der Gewinn bei einem Sieg auf der i-ten Stufen beträgt \(\gamma x_{i}\)) und \(\alpha_{1}, \alpha_{2}, \cdots,\alpha_{n}\) die Steigerungsfaktoren (d.h. \(\alpha_{i}x_{i}=x_{i+1}\)).\\\\
			D.h. der Gewinn auf der i-ten Stufe entspricht entweder \(-x_{i}\) oder \(\gamma x_{i}\). Sei also \(\psi(p_{i})\) dieser Gewinn. Dann gilt fuer den Erwartungswert
			$$
				E(X)=\sum_{i=1}^n \psi(p_{i})p_{i}
			$$
			Und fuer jeden Einsatz gilt
			$$
				x_{k}=\alpha_{k-1}x_{k-1}=\alpha_{k-1}\alpha_{k-2}x_{k-2}=\cdots=\prod_{i=1}^{k-1}\alpha_{i}x_{1}
			$$
			Sei \(\rho\) der maximale zur Verfuegung stehende Betrag, d.h
			$$
				\rho \geq \sum_{i=1}^{n}x_{i}=\sum_{i=1}^{n}\prod_{k=1}^{i-1}\alpha_{k}x_{1}
			$$
			Das Ziel ist jetzt also den Erwartungswert nach den Einsaetzen zu optimieren und nach Strategien in Abhaengigkeit von den Gewinnwahrscheinlichkeiten zu finden.\\
			Berechne zunaechst also den Gradienten von E
			$$
				(\nabla E)_{i}=\sum_{k=1}^{n}\frac{d\psi(p_{k})p_{k}}{dx_{i}}=\frac{d\psi(p_{i})p_{i}}{dx_{i}}
			$$
			Falls \(\psi(p_{i})=-x_{i}\) gilt
			\begin{center}
				\[(\nabla E)_{i}=\frac{d\psi(p_{i})p_{i}}{dx_{i}}=-\frac{dx_{i}p_{i}}{dx_{i}}=-p_{i}\]
			\end{center}
			und falls \(\psi(p_{i})=\gamma x_{i}\) gilt
			\begin{center}
				\[(\nabla E)_{i}=\frac{d\psi(p_{i})p_{i}}{dx_{i}}=\frac{d\gamma x_{i}p_{i}}{dx_{i}}=\gamma p_{i}\]
			\end{center}
			Sei also im Folgendem
			\begin{center}
				\[ \xi(i) :=
					\begin{cases}
						-1, & \psi(p_{i})=-x_{i} \\
						\gamma, & \psi(p_{i})=\gamma x_{i}
					\end{cases}
				\]
			\end{center}
			mit dem Gradienten des Erwartungswerts
			\begin{center}
				\[(\nabla E)_{i}=\xi(i)p_{i}\]
			\end{center}
			Sei nun \(\omega\) die Norm von X, dann gilt fuer die Nebenbedingung
			\[g(X):=\sum_{i=1}^{n}x_{i}^2-{\omega}^2\]
			Dann gilt fuer \[(\nabla \lambda g)_{i} = 2\lambda x_{i}\]\\
			Es laesst sich also mit der Methode der Lagrange-Multiplikatoren die folgende Funktion aufstellen \[\Lambda(X,\lambda) := E(X)+\lambda g(X)\]
			\[\nabla \Lambda(X,\lambda) = \nabla E(X) + \nabla \lambda g(X)\]
			Durch die bereits berechneten Gradienten folgt
			\[(\nabla \Lambda (X,\lambda))_{k}=
				\begin{cases}
					\xi (k)p_{k}+2\lambda x_{k}, & 1\leq k\leq n\\
					\sum_{i=1}^{n}x_{i}^2-\omega^2, & k = n+1
				\end{cases}\]
			und es muss gelten das
			\[\nabla \Lambda(X,\lambda)=0\]
			Daher folgt
			\[\lambda x_{i} = -\frac{\xi(i)p_{i}}{2}\]
			Durch die Methode der Lagrange-Multiplikatoren koennen wir annehmen das \(\lambda \neq 0\)
			\[x_{i}= -\frac{\xi(i)p_{i}}{2\lambda}\]
			Jetzt kann jedes x in die letzte Gleichung eingesetzt werden
			\[\sum_{i=1}^{n}(-\frac{\xi(i)p_{i}}{2\lambda})^2-\omega^2=0\]
			\[\sum_{i=1}^{n}\frac{\xi(i)^2p_{i}^2}{4\lambda^2}=\omega^2\]
			\[\sum_{i=1}^{n}\xi(i)^2p_{i}^2=4\omega^2\lambda^2\]
			\[\lambda = \pm\frac{\sqrt{\sum_{i=1}^{n}\xi(i)^2p_{i}^2}}{2\omega}\]
			Eingesetzt in die anderen n Gleichungen folgt
			\[x_{i}=\mp\frac{\omega\xi(i)p_{i}}{\sqrt{\sum_{k=1}^{n}\xi(k)^2p_{k}^2}}\]
			Sei nun o.B.d.A \(p_{i}\geq0,5\), d.h es wird immer auf den besseren Ausgang gesetzt. Dann muessen alle \(x_{i}\) positiv sein. Daher ist
			$$
			x_{i}=\frac{\gamma\omega p_{i}}{\sqrt{\sum_{k=1}^{n}\gamma^2p_{k}^2}}
			=\frac{\omega p_{i}}{\sqrt{\sum_{k=1}^{n}p_{k}^2}}
			$$
			mit positivem \(\lambda\) oder
			$$
			x_{i}=\frac{\omega p_{i}}{\sqrt{\sum_{k=1}^{n}p_{k}^2}}
			$$
			mit negativem \(\lambda\). \\
			Unter der Annahme das fuer alle i,j gilt, dass \(p_{i}=p_{j}\) gilt sogar
			$$
			x_{i}=\frac{\omega p_{i}}{\sqrt{\sum_{k=1}^{n}p_{k}^2}}=\frac{\omega p_{i}}{\sqrt{\sum_{k=1}^{n}p_{i}^2}}=\frac{\omega }{\sqrt{\sum_{k=1}^{n}1}}=\frac{\sqrt{2}\omega}{\sqrt{n(n+1)}}
			$$\\
			Berechne nun also die geraenderte Hesse-Matrix:
			$$
			H_{L}(X,\lambda)=
			\left(
			\begin{array}{ccccc}
				0 & 2x_{1} & 2x_{2} & \cdots & 2x_{n} \\
				2x_{1} & 2\lambda & 0 & \cdots & 0 \\
				2x_{2} & 0 & \ddots & \ddots & \vdots \\
				\vdots & \vdots & \ddots & \ddots & 0 \\
				2x_{n} &  0 & \cdots & 0 & 2\lambda
			\end{array}
			\right)
			$$
			Um zu zeigen das die berechneten Werte Maxima sind, ist zu zeigen,\\ dass \(det(H_{L}(X,\lambda))<0\).\\
			Zeige also zunächst mit vollstaendiger Induktion nach n, dass
			$$
			det(H_{L}(X,\lambda))=-2^n\lambda^{n-2}\sum_{i=1}^{n}x_{i}^2
			$$
			fuer alle \(n\geq2\) gilt.\\
			\(n=2\):
			$$
			|H_{L}(X,\lambda)|=
			\begin{vmatrix}
				0 & 2x_{1} \\
				2x_{1} & 2\lambda\\
			\end{vmatrix}
			=-4x_{1}^2=-2^n\lambda^{n-2}x_{1}^2
			$$
			Induktionsschritt:
			$$
			|H_{L}(X,\lambda)|=
			\begin{vmatrix}
				0 & 2x_{1} & 2x_{2} & \cdots & 2x_{n} \\
				2x_{1} & 2\lambda & 0 & \cdots & 0 \\
				2x_{2} & 0 & \ddots & \ddots & \vdots \\
				\vdots & \vdots & \ddots & \ddots & 0 \\
				2x_{n} &  0 & \cdots & 0 & 2\lambda
			\end{vmatrix}
			=
			$$
			$$
			(-1)^{2(n+1)}2\lambda
			\begin{vmatrix}
			0 & 2x_{1} & 2x_{2} & \cdots & 2x_{n-1} \\
			2x_{1} & 2\lambda & 0 & \cdots & 0 \\
			2x_{2} & 0 & \ddots & \ddots & \vdots \\
			\vdots & \vdots & \ddots & \ddots & 0 \\
			2x_{n-1} &  0 & \cdots & 0 & 2\lambda
			\end{vmatrix}
			+(-1)^{n}2x_{n}
			\begin{vmatrix}
			2x_{1} & 2\lambda & 0 & \cdots & 0 \\
			2x_{2} & 0 & \ddots & \ddots & \vdots \\
			\vdots & \vdots & \ddots & \ddots & 0 \\
			\vdots & 0 & \cdots & 0 & 2\lambda \\
			2x_{n} &  0 & \cdots & \cdots & 0
			\end{vmatrix}
			$$
			Fuer die erste Determinante lässt sich nun die Induktionsvoraussetzung einsetzen und die zweite Determinante laesst sich einmal nach der letzten Zeile und dann wiederholt nach der zweiten Spalte entwickeln. Dadurch entsteht einmal der Faktor \(2x_{n}\), \((n-2)\)-mal der Faktor \(2\lambda\) und einmal \((-1)^{n-1}\).
			$$
			|H_{L}(X,\lambda)|=(-1)^{2(n+1)}2\lambda(-2^{n-1}\lambda^{n-3}\sum_{i=1}^{n-1}x_{i})+(-1)^{n}2x_{n}2x_{n}(2\lambda)^{n-2}(-1)^{n-1}
			$$
			$$
			=-2^{n}\lambda^{n-2}\sum_{i=1}^{n-1}x_{i}+(-1)^{2n-1}2^{n}\lambda^{n-2}x_{i}^{2}
			=-2^n\lambda^{n-2}\sum_{i=1}^{n}x_{i}^2
			$$
			$\hfill\square$\\\\
			Daraus folgt, dass die Punkte genau dann ein lokales Maximum sind, wenn \(\lambda>0\). Für gerade n sind die Punkte unabhängig von \(\lambda\) immer ein lokales Maximum. Wie bereits gezeigt gilt, falls \(\lambda>0\)
			$$
			x_{i}=\frac{\omega p_{i}}{\sqrt{\sum_{k=1}^{n}p_{k}^2}}
			$$
\end{document}