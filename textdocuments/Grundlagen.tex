\documentclass[11pt]{scrartcl}
\usepackage{german}
\begin{document}
	\section{Ziel der Arbeit}
		Die vorliegende Arbeit verfolgt im wesentlichen zwei Ziele: Zum einen soll ein B\"orsenkurs m\"oglichst gut simuliert werden k\"onnen, zum anderen sollen Strategien zum effizienten Wetten auf einen B\"orsenkurs entwickelt werden. Dabei sollen unterschiedliche Optimierungen m\"oglich sein (z.B. minimales Risiko, maximaler Gewinn). Diese Strategien k\"onnen sowohl theoretisch, als auch auf Basis der aus der Simulation gewonnenen Erkenntnisse erarbeitet werden.
	\section{Datenauswahl}
		Um ein Gef\"uhl f\"ur B\"orsenkurse zu bekommen und darauf aufbauen die Simulation zu entwickeln, wurde zun\"achst ein Kurs als beispielhaftes Untersuchungsobjekt ausgew\"ahlt.
		Dieser Kurs sollte folgende Anforderungen erf\"ullen:
		\begin{itemize}
			\item Die Kursdaten sollten \"uber einen m\"oglichst grossen Zeitraum verf\"ugbar sein.
			\item Die Kursdaten sollten als min\"utliche Trades verf\"ugbar sein.
			\item Der Kurs sollte \"uber eine gewisse Repr\"asentativit\"at verf\"ugen.
			\item Der Kurs sollte im grossen und ganzen einen stetigen Verlauf haben.
		\end{itemize}
		Auf der Basis dieser Anforderungen haben wir den Goldpreis in US-Dollar als anf\"angliches Untersuchungsobjekt ausgew\"ahlt. Wir haben daf\"ur die Kursdaten \"uber etwa 10 Jahre von 2007 bis 2016 in einer Datei zusammengef\"uhrt und nach einem einheitlichen Format angeordnet.
\end{document}