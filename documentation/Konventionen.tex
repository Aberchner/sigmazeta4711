\documentclass[11pt]{scrartcl}
\usepackage{german}
\begin{document}
	\section{Sprachkonventionen:}
		Sprachkonventionen:
		\begin{itemize}
			\item Variablen auf Englisch
			\item Kommentare auf Englisch
			\item Dokumentation auf Deutsch
			\item Textdokumente auf Deutsch
		\end{itemize}
		Dateinamen:
		\begin{itemize}
			\item Quelltextdateien auf Englisch benennen
			\item Datendateien auf Englisch benennen
			\item Dokumentationsdateien auf Deutsch benennen
			\item Textdokumente auf Deutsch benennen
		\end{itemize}
	\section{Datenkonventionen}
		Damit die Programme trotz unterschiedlicher Datenquellen universal geschrieben werden können, legen wir folgendes Datenformat f\"ur jeden einzelnen Trade fest:
		\begin{itemize}
			\item[] (JJJJMMTT, HHMMSS, Startwert, Endwert, Höchstwert, Tiefstwert)
		\end{itemize}
		Dabei ist das erste Feld das Datum und das zweite Feld die Uhrzeit.
	\section{Coding Konventionen}
		\subsection{Maximale Zeilenlänge}
			Maximal 79 Zeichen pro Zeile (''Linie'' von Spyder)!
		\subsection{Leerzeilen}
			\begin{itemize}
				\item[] 2 Leerzeilen vor und nach Klassen und wichtigen Funktionen.
				\item[] 1 Leerzeile vor und nach sonstigen Funktionen.
				\item[] Extra Leerzeilen zum Gruppieren von zusammengeh\"origen Funktionen.
				\item[] Bei Bedarf: Leerzeilen in Funktionen zur besseren Verst\"andlichkeit.
			\end{itemize}
		\subsection{Kommentare}
			\textbf{Aussagekr\"aftige Kommentare!!!}
			\begin{itemize}
				\item[] Vor jeder Funktion und Klasse kurze Beschreibung was passiert in einem Blockkommentar.
				\item[] Bei l\"angeren Blockkommentaren Abs\"atze durch Zeilen mit einem ''\#'' bilden.
				\item[] Kommentare in der selben Zeile, wie der Quelltext (Inline Kommentare) in Ausnahmef\"allen benutzen, wenn die Aufgabe der Codezeile nicht offensichtlich ist.
			\end{itemize}
		\subsection{Namensgebung}
			module\_name, package\_name, ClassName, method\_name, 
			ExceptionName, function\_name, GLOBAL\_CONSTANT\_NAME, global\_var\_name, instance\_var\_name, function\_parameter\_name, local\_var\_name
\end{document}