\documentclass[11pt]{scrartcl}
\begin{document}
	\section{Sprachkonventionen:}
		Sprachkonventionen:
		\begin{itemize}
			\item Variablen auf Englisch
			\item Kommentare auf Englisch
			\item Dokumentation auf Deutsch
			\item Textdokumente auf Deutsch
		\end{itemize}
		Dateinamen:
		\begin{itemize}
			\item Quelltextdateien auf Englisch benennen
			\item Datendateien auf Englisch benennen
			\item Dokumentationsdateien auf Deutsch benennen
			\item Textdokumente auf Deutsch benennen
		\end{itemize}
	\section{Datenkonventionen}
	\section{Coding Konventionen}
		\subsection{Maximale Zeilenlänge}
			Maximal 79 Zeichen pro Zeile ("Linie" von Spyder)!
		\subsection{Leerzeilen}
			\begin{itemize}
				\item[] 2 Leerzeilen vor und nach Klassen und wichtigen Funktionen.
				\item[] 1 Leerzeile vor und nach sonstigen Funktionen.
				\item[] Extra Leerzeilen zum Gruppieren von zusammengehörigen Funktionen.
				\item[] Bei Bedarf: Leerzeilen in Funktionen zur besseren Verständlichkeit.
			\end{itemize}
		\subsection{Kommentare}
			\textbf{Aussagekraeftige Kommentare!!!}
			\begin{itemize}
				\item[] Vor jeder Funktion und Klasse kurze Beschreibung was passiert in einem Blockkommentar.
				\item[] Bei längeren Blockkommentaren Absätze durch Zeilen mit einem "\#" bilden.
				\item[] Kommentare in der selben Zeile, wie der Quelltext (Inline Kommentare) in Ausnahmefällen benutzen, wenn die Aufgabe der Codezeile nicht offensichtlich ist.
			\end{itemize}
		\subsection{Namensgebung}
			\begin{itemize}
				\item Module:
				\item[] kurze, kleingeschriebene Namen, Unterstriche bei Bedarf
				\item Packages:
				\item[] kurze, kleingeschriebene Namen, keine Unterstriche
				\item Klassen:
				\item[] CamelCases
				\item Funktionen:
				\item[] kleingeschriebene Namen, Woerter durch Unterstriche getrennt
				\item Konstanten:
				\item[] großgeschriebene Namen
			\end{itemize}
			
\end{document}